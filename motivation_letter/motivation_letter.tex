%------------------------------------------------------------------------------
% A customized version of Vel's interpretation of the...
% Freeman Curriculum Vitae
% XeLaTeX Template
% Version 3.0 (September 3, 2021)
% see: https://www.LaTeXTemplates.com
% License:
% CC BY-NC-SA 4.0 (https://creativecommons.org/licenses/by-nc-sa/4.0/)
%------------------------------------------------------------------------------

%------------------------------------------------------------------------------
% Preamble
%------------------------------------------------------------------------------
\documentclass[11pt]{FreemanML}

% Widths of the two columns, specified here as a ratio summing to 1 to
% correspond to percentages; adjust as needed for your content 
\columnratio{0.55, 0.45} 

\usepackage{parskip}
\usepackage{csquotes}
\usepackage{lettrine}
\usepackage{framed}
\usepackage[singlespacing]{setspace}
%\usepackage{graphicx}
%\usepackage[export]{adjustbox}

% Headers and footers can be added with the following commands: \lhead{},
% \rhead{}, \lfoot{} and \rfoot{}

\lfoot{\footnotesize{The \LaTeX code to reproduce the motivation letter is available at: \url{https://github.com/sbissantz/smip_application}.}}

%------------------------------------------------------------------------------

\begin{document}

\thispagestyle{empty}

\begin{paracol}{2} % Begin two-column mode

%------------------------------------------------------------------------------
%	YOUR NAME AND CURRICULUM VITAE TITLE
%----------------------------------------------------------------------------------------

\parbox[][0.11\textheight][c]{\linewidth}{ 
	% Box to hold your name and CV title; change the fixed height as needed to
	% match the colored box to the right
	\centering % Horizontally center text
	% Your name
	{\Huge\textcolor{headings}{Letter of Motivation}} 

	% AB-line
	\par\noindent\rule{6cm}{0.4pt}
	\medskip 

	% Vertical whitespace
	{\sffamily\Large{Steven Marcel Bißantz}}
	%{\cursivefont\Huge\textcolor{headings}{Curriculum Vitae}}
	
	\vfill % Push content to the top of the box
}

\switchcolumn % Switch to the second (right) column

%------------------------------------------------------------------------------
%	COLORED CONTACT DETAILS BOX
%------------------------------------------------------------------------------

\parbox[top][0.11\textheight][c]{\linewidth}{ % Box to hold the colored box; change the fixed height as needed to match the box to the left
	\colorbox{shade}{ % Create colored box and specify background color
		% Start a table with two columns, the table will ensure everything is
		% aligned
		\begin{supertabular}{@{\hspace{3pt}} p{0.05\linewidth} | p{0.775\linewidth}}
			\raisebox{-1pt}{\faHome} & Nordparkstraße 10b, 76829 Landau \\ % Address
			\raisebox{-1pt}{\faPhone} & +49 1525 3606062\\ % Phone number
			\raisebox{-1pt}{\small\faEnvelope} & \href{mailto:bissantz@uni-landau.de}{bissantz@uni-landau.de} \\ % Email address
			\raisebox{-1pt}{\small\faGithub} & \href{https://github.com/sbissantz}{https://github.com/sbissantz} \\ % Website
			%\raisebox{-1pt}{\faLinkedinSquare} & \href{https://www.linkedin.com/in/username}{https://www.linkedin.com/in/username} \\ % LinkedIn profile
			% See fontawesome.pdf in the Fonts folder for all icons you can use
		\end{supertabular}
	}
	\vfill % Push content to the top of the box
}

%------------------------------------------------------------------------------
%	COLORED CONTACT DETAILS BOX
%------------------------------------------------------------------------------

\switchcolumn % Switch to the second (right) column

% Red color heading as font color
\color{headings} 
 % Extra whitespace before the next section
\vspace{2cm}
\underline{\textbf{\footnotesize{Steven Bißantz $\cdot$ Nordparkstraße 10b $\cdot$ 76829 Landau}}}

\color{text}
\bigskip
\begin{leftbar}
University of Mannheim \newline
Research Training Group SMiP \newline
B6, 30-32 \newline
D-68159 Mannheim
\end{leftbar}

\switchcolumn

\vspace{5cm}
\begin{flushright}
Landau, \today
\end{flushright}
\end{paracol} % End two-column mode

\vspace{1cm}
\section*{My Motivation to become a SMiP-Associate}
\bigskip
Dear members of the selection committee,
\smallskip

\onehalfspacing

\lettrine[lines=3]{I}{}n 2005 John Ioannidis proclaimed \enquote{Why Most
Published Research Findings Are False}. Meanwhile, luminaries like Brian Nosek
suggest that one driving force towards improving science could be a profound
statistical (re-)education. SMiP fosters this transformation. The program
stands right in the middle, providing (a) a coherent research network, (b) a
cognitively stimulating environment, and, most importantly, (c) the proficient
staff to guide the next generation of research aspirants. I want to be part of
it. Why? Because I enjoy developing at my own best. A structured Ph.D. program
will help me to do so. For example, SMiP's \textit{foundations} will aid in
developing essential soft and hard skills for my scientific career, like
proficiency in scientific writing and in-depth know-how on quantitative
methods. The \textit{extensions} will put the final touch on my methodological
training. However, they will also allow me to sharpen my profile, which I'll
explain next. 


\begin{color}{headings}
	\begin{center}
		Metascience $\cdot$ Quantitative Methods $\cdot$ Computer Science
	\end{center}
\end{color}
\vspace{-0.2cm}

\paragraph{Metascience}
Thinking outside the box has always been prior for me. So despite the main
scope of my schedule, like social psychology (emphasis: stereotypes, racism \&
discrimination) and empirical research methods (emphasis: multivariate analysis
methods), I attended additional courses in philosophy (emphasis: logic \&
philosophy of science). The cross-disciplinary toolkit now intertwines with my
doctoral studies on metascience. In October 2021, Prof. Dr. Eunike Wetzel
recruited me for her team in the DFG Priority Program
\textsc{meta-rep}\footnote{\url{https://www.psy.lmu.de/soz/meta-rep/index.html}}.
I do meta-research at her lab on the role of measurement in the replicability
of empirical findings. Next summer, for example, we will conduct a
cross-disciplinary, multisample replication project. Incorporating an
experiment on the influence of measurement on replicability and the
heterogeneity of effect sizes abstracts one of our three projects. My team
member, Caroline Boehm, told me that Prof. Dr. Beatrice G. Kuhlmann is also a
vibrant meta scientist. Her advice throughout the program would be perfect
guidance.

\paragraph{Quantitative Methods}

As an undergraduate student, I became passionate about quantitative methods.
Besides introductory courses on data collection, descriptive statistics, and
multivariate analysis, I acquired advanced skills in self-studies. For example,
I trained myself in Bayesian Data Analysis and started lately with
probabilistic machine learning. My methodological skill set did pay off in 2020
during my research internship at the Center for Educational Research
(\textsc{zepf}). Dr. Inga Wagner invited me to join her team, elucidating mode
effects in comparative studies (\textsc{vera}). Our efforts finally led to an
article we could publish in the European Journal of Psychology of Education in
2021. All offered SMiP courses will support me going forward with my modeling.
But as a committed Bayesian, I am curious about working with Prof. Dr. Jeffrey
N. Rouder.

\paragraph{Computer Science}

Developing computer skills have ever played an essential role in my life. Going
fully open-source -- switching to (fedora) Linux as an undergraduate -- is a
good example. Acquiring coding skills is another. As a master's student, I
started R programming. It became the kickoff for developing \textit{elisr}.
\enquote{Exploratory Likert Scaling} is an R package for exploratory
dimensionality analysis in the social sciences, recently published on CRAN. But
starting my PhD, I wanted to go further. To gain flexibility in statistical
modeling, I have lately put my hands on the probabilistic programming language
\textit{Stan}. The goal is to employ it as the go-to analysis tool throughout
my PhD. As far as I know, SMiP does not yet offer programming courses in Stan.
But this could be a superb chance for contribution. Feeding first-hand
information into the network, preparing a Stan webinar, for example, might be a
place for me to start. Besides, I am fascinated with Git. Having
back-and-forths about version control at SMiP can foster
open-science/code/source practices and might be a sound argument for having me.

\medskip

Thank you for considering my application.

\medskip

Sincerely yours,

\bigskip
\bigskip

%------------------------------------------------------------------------------
% Signature
%------------------------------------------------------------------------------

\begin{tabular}{p{8cm}p{.5cm}l}
\dotfill \\ 
\centering{Steven Marcel Bißantz} 
\end{tabular}% 
\vfill

%----------------------------------------------------------------------------------------

\end{document}
