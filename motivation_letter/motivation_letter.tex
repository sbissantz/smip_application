%------------------------------------------------------------------------------
% A customized version of Vel's interpretation of the...
% Freeman Curriculum Vitae
% XeLaTeX Template
% Version 3.0 (September 3, 2021)
% see: https://www.LaTeXTemplates.com
% License:
% CC BY-NC-SA 4.0 (https://creativecommons.org/licenses/by-nc-sa/4.0/)
%------------------------------------------------------------------------------

%------------------------------------------------------------------------------
% Preamble
%------------------------------------------------------------------------------
\documentclass[10pt]{FreemanML}

% Widths of the two columns, specified here as a ratio summing to 1 to
% correspond to percentages; adjust as needed for your content 
\columnratio{0.55, 0.45} 

\usepackage{csquotes}
\usepackage{lettrine}
\usepackage{framed}
\usepackage[singlespacing]{setspace}
%\usepackage{graphicx}
%\usepackage[export]{adjustbox}

% Headers and footers can be added with the following commands: \lhead{},
% \rhead{}, \lfoot{} and \rfoot{}

\lfoot{\footnotesize{The \LaTeX code to reproduce the CV and a free PDF Version is available at: \url{https://github.com/sbissantz/smip_application/curriculum_vitae}.}}

%------------------------------------------------------------------------------

\begin{document}

\begin{paracol}{2} % Begin two-column mode

%------------------------------------------------------------------------------
%	YOUR NAME AND CURRICULUM VITAE TITLE
%----------------------------------------------------------------------------------------

\parbox[][0.11\textheight][c]{\linewidth}{ 
	% Box to hold your name and CV title; change the fixed height as needed to
	% match the colored box to the right
	\centering % Horizontally center text
	% Your name
	{\Huge\textcolor{headings}{Curriculum Vitae}} 

	% AB-line
	\par\noindent\rule{6cm}{0.4pt}
	\medskip 

	% Vertical whitespace
	{\sffamily\Large{Steven Marcel Bißantz}}
	%{\cursivefont\Huge\textcolor{headings}{Curriculum Vitae}}
	
	\vfill % Push content to the top of the box
}

\switchcolumn % Switch to the second (right) column

%------------------------------------------------------------------------------
%	COLORED CONTACT DETAILS BOX
%------------------------------------------------------------------------------

\parbox[top][0.11\textheight][c]{\linewidth}{ % Box to hold the colored box; change the fixed height as needed to match the box to the left
	\colorbox{shade}{ % Create colored box and specify background color
		% Start a table with two columns, the table will ensure everything is
		% aligned
		\begin{supertabular}{@{\hspace{3pt}} p{0.05\linewidth} | p{0.775\linewidth}}
			\raisebox{-1pt}{\faHome} & Nordparkstraße 10b, 76829 Landau \\ % Address
			\raisebox{-1pt}{\faPhone} & +49 1525 3606062\\ % Phone number
			\raisebox{-1pt}{\small\faEnvelope} & \href{mailto:bissantz@uni-landau.de}{bissantz@uni-landau.de} \\ % Email address
			\raisebox{-1pt}{\small\faGithub} & \href{https://github.com/sbissantz}{https://github.com/sbissantz} \\ % Website
			%\raisebox{-1pt}{\faGithub} & \href{https://github.com/username}{https://github.com/username} \\ % GitHub profile
			%\raisebox{-1pt}{\faLinkedinSquare} & \href{https://www.linkedin.com/in/username}{https://www.linkedin.com/in/username} \\ % LinkedIn profile
			% See fontawesome.pdf in the Fonts folder for all icons you can use
		\end{supertabular}
	}
	\vfill % Push content to the top of the box
}

%------------------------------------------------------------------------------
%	COLORED CONTACT DETAILS BOX
%------------------------------------------------------------------------------

\switchcolumn % Switch to the second (right) column

\color{headings} %Rote Farbe aus den Überschriften im Lebenslauf
\vspace{2cm} % Extra whitespace before the next section
\underline{\textbf{\footnotesize{Steven Bißantz $\cdot$ Nordparkstraße 10b $\cdot$ 76829 Landau}}}

\color{text}
\bigskip
\begin{leftbar}
Prof. Dr. Michaela Maier \newline
Fortstraße 7; K 5.26  \newline
76829 Landau 
\end{leftbar}

\switchcolumn

\vspace{5cm}
\begin{flushright}
Landau, \today
\end{flushright}
\end{paracol} % End two-column mode

\vspace{1cm}
\section*{My Motivation to become a SMiP-Associate}
\bigskip
Sehr geehrte Frau Prof. Dr. Maier, 
\medskip

%\renewcommand{\baselinestretch}{1.4}\normalsize

\onehalfspacing

\enquote{Most published research is false!}, John Ioannidis stated in 2005.
Meanwhile, luminaries like Brian Nosek suggest that one driving force towards
improving science could be a profound statistical (re-)education. SMiP fosters
this cultural transformation. The program stands right in the middle, providing
(a) a coherent research network, (b) a cognitively stimulating environment,
and, most importantly, the proficient staff to guide the next generation of
research aspirants. I want to be part of it. Why? Because I want to develop at
my own best. The structured Ph.D. program will help me to do so. For example,
the \enquote{foundations} will aid in developing essential soft and hard skills
for my scientific career, like proficiency in scientific writing and in-depth
know-how on quantitative methods. The \enquote{extensions} will put the
finishing touch on my methodological training. However, they will also allow me
to sharpen my profile, which I'll explain next. 

\subsection{Metascience $\cdot$ Quantitative Methoden $\cdot$ Datenwissenschaften}

\paragraph{Metascience}

Thinking outside the box has always been prior for me. So despite the main
scope of my schedule, like social psychology (emphasis: stereotypes, racism &
discrimination) and empirical research methods (emphasis: multivariate analysis
methods), I attended additional courses in philosophy (emphasis: logic \&
philosophy of science). The cross-disciplinary toolkit now intertwines with my
doctoral studies on metascience. In October 2021, Prof. Dr. Eunike Wetzel
recruited me for her team in the DFG Priority Program META-REP *LINK*. Now, I
do meta-research at her lab on the role of measurement in the replicability of
empirical findings. Next summer, for example, we will conduct a
cross-disciplinary, multisample replication project. Combining it with an
experiment on the influence of measurement on replicability and effect size
heterogeneity abstracts one of our three projects. My team member, Caroline
Boehm, told me that Prof. Dr. Beatrice Kuhlmann is also passionate about
metascience. She might be interested in becoming my secondary supervisor.

\paragraph{Quantitative Methods}

As an undergraduate student, I became passionate about quantitative methods.
Besides introductory courses on data collection, descriptive statistics, and
multivariate analysis, I acquired advanced skills in self-studies. For example,
I trained myself in Bayesian Data Analysis and started lately with
probabilistic machine learning. My methodological skill set did pay off in 2020
during my research internship at the Center for (\textsc{zepf}). Dr. Inga
Wagner invited me to join her team, elucidating mode effects in comparative
studies (VERA). Our efforts finally led to an article we could publish in the
European Journal of Psychology of Education in 2021. All SMiP courses offered
will support me going forward in my modeling. But as a committed Bayesian, I am
curious about working with Prof. Dr. Jeffrey N. Rouder. He might also be
interested in becoming my third supervisor.

\paragraph{Computer Science}

Data science In Mannheim, I first enthusiastically learned how to use SPSS and
Stata; in the meantime, however, more complex estimation and data problems in
my work at the zepf also require the experienced use of the statistical
software R (e.g., in dealing with multilevel models). Furthermore, I already
learned to program in this language in Bamberg. An interesting project in this
context was the development of elisr. "Exploratory Likert Scaling in R" is an
R package for exploratory dimensional analysis in the social sciences, which I
was recently able to publish on CRAN. In my master's thesis ("Exploratory
Dimensionality Analysis in the Social Sciences"), I am currently investigating
the advantages of this method in finding latent variables in complex data sets
(e.g., in comparison with exploratory factor analysis)


for your interest in a structured Ph.D. program at the Research Training Group
SMiP. Elaborate on your study plans and your research interests. It is very
important to show that your research interests meet closely with the current
research of the respective  faculty member(s) you would be interested in
working with.The motivation letter should be in English and not exceed two
pages (not more than 700 words).


%------------------------------------------------------------------------------
% Signature
%------------------------------------------------------------------------------
\vfill
\begin{tabular}{p{8cm}p{.5cm}l}
\dotfill \\ 
\centering{Steven Bißantz} 
\end{tabular}% 
\vfill

%----------------------------------------------------------------------------------------

\end{document}
