%------------------------------------------------------------------------------
% A customized version of Vel's interpretation of the...
% Freeman Curriculum Vitae
% XeLaTeX Template
% Version 3.0 (September 3, 2021)
% see: https://www.LaTeXTemplates.com
% License:
% CC BY-NC-SA 4.0 (https://creativecommons.org/licenses/by-nc-sa/4.0/)
%------------------------------------------------------------------------------

%------------------------------------------------------------------------------
% Preamble
%------------------------------------------------------------------------------
\documentclass[10pt]{FreemanCV}

% Widths of the two columns, specified here as a ratio summing to 1 to
% correspond to percentages; adjust as needed for your content 
\columnratio{0.55, 0.45} 

\usepackage{csquotes}

% Headers and footers can be added with the following commands: \lhead{},
% \rhead{}, \lfoot{} and \rfoot{}
\lfoot{{\sffamily For the \LaTeX code and the pdf-Version, please see:
\url{https://github.com/sbissantz/smip_application}}}

%------------------------------------------------------------------------------

\begin{document}

\begin{paracol}{2} % Begin two-column mode

%------------------------------------------------------------------------------
%	YOUR NAME AND CURRICULUM VITAE TITLE
%----------------------------------------------------------------------------------------

\parbox[][0.11\textheight][c]{\linewidth}{ 
	% Box to hold your name and CV title; change the fixed height as needed to
	% match the colored box to the right
	\centering % Horizontally center text
	
	{\Huge\textcolor{headings}{Curriculum Vitae}} % Your name
	\par\noindent\rule{6cm}{0.4pt}
	\medskip % Vertical whitespace
	
	{\sffamily\Large{Steven Marcel Bißantz}}
	%{\cursivefont\Huge\textcolor{headings}{Curriculum Vitae}}
	
	\vfill % Push content to the top of the box
}

%------------------------------------------------------------------------------
%	MAJOR RESEARCH PROJECT
%------------------------------------------------------------------------------

\section{Doctoral Research}

{\raggedright\textbf{Measurement \& Replicability in the Sciences --
Meta-Scientific Investigations in The Role of Measurement in The Replicability
of Empirical Findings}\par}

\medskip % Vertical whitespace

My research concentrates on the role of measurement in replicability. In
particular, the focus is on how (a) modifying validated item-based measurements
or (b) the practice of tossing various items together in a \enquote{scale}
(i.e., omitting the validation process) affects the replicative success of a
study. Despite the replicative success, I illuminate the influence of these
ad-hoc and post-hoc scales on effect size estimates and their induced
heterogeneity.

\medskip % Extra vertical whitespace before the next section

%------------------------------------------------------------------------------
%	WORK EXPERIENCE
%------------------------------------------------------------------------------

\section{Work Experience}

% Each job is added with a \jobentry command. Below is an empty one to use as a template:

%\jobentry
%	{} % Duration
%	{} % FT/PT (full time or part time)
%	{} % Employer
%	{} % Job title
%	{} % Description

% All 5 parameters must be supplied but any can be empty if you don't need them

%------------------------------------------------

\jobentry
	{Current, from Oct 2021} 
	{} 
	{University of Koblenz-Landau \hfill{\footnotesize{\textsc{(Prof. Dr. Eunike Wetzel)}}}}
	{PhD Student, Psychology}
	{Mainly, I am involved in meta-science; learn about quantitative methods, and teach test theory, diagnostics, and R.}

%------------------------------------------------

\jobentry
	{May 2019 -- Jul 2021}
	{} 
	{Centre of Educational Research
	\hfill\footnotesize{\textsc{(ZEPF)}}}
	{Research Assistent, Psychology}  
	{I researched mode effects in performance tests (\textsc{VERA}) using structural equation models and multilevel modeling.}

%------------------------------------------------

\jobentry
	{Sep 2016 -- Apr 2018}
	{} 
	{University of Mannheim
	\hfill\footnotesize{\textsc{(Prof. Dr. Irena Kogan)}}}
	{Research Assistent, Sociology}
	{Data analysis, literature research, and the administration of the internet
	presence in English and German were among my tasks.}

%------------------------------------------------

\jobentry
	{Sep 2016 -- Feb 2017}
	{} 
	{University of Mannheim
	\hfill\footnotesize{\textsc{(Prof. Dr. Frank Kalter)}}}
	{Research Assistent, Sociology}
	{As a tutor, I reviewed the course material content, clarified unanswered
	questions, and deepened the understanding of my fellow students in
complementary teaching sessions.}

%------------------------------------------------

\jobentry
	{Apr 2015 -- Jul 2015}
	{} 
	{Mannheim Centre for European Social Research
	\hfill\footnotesize{\textsc{(MZES)}}}
	{Research Assistent, Sociology}  
	{After theoretical and practical training, I conducted computer-assisted
	telephone interviews (CATI).}

%----------------------------------------------------------------------------------------
% Memberships	
%----------------------------------------------------------------------------------------

\section{Memberships}

\jobentry
	{Current, from Feb 2021} 
	{}
	{University of Koblenz-Landau}
	{Member of the Open Science Comission}  
	{At the OSC we promote the implementation of open science practices.}

%----------------------------------------------------------------------------------------
%	REFERENCES
%----------------------------------------------------------------------------------------

%\section{References}

%\textit{References available on request} % Uncomment if you'd rather not
%include references and remove the section below

%------------------------------------------------

% This section is laid out using a table. A \tableentry command adds lines with the following parameters:

%\tableentry{Heading}{Content}{spaceafter}
% All 3 parameters must be supplied but any can be empty if you don't need them
% A "spaceafter" value in the third parameter will add some vertical space -- this is to be used between headings, leave it empty for no extra space

%------------------------------------------------

%\begin{supertabular}{r l} 
	
	%\tableentry{}{\textbf{Dr. Isaac Kleiner}}{spaceafter}
	%\tableentry{Position}{Professor}{}
	%\tableentry{Employer}{\href{https://web.mit.edu/physics/}{Department of Physics}}{}
	%\tableentry{}{\href{https://web.mit.edu}{\textit{Massachusetts Institute of Technology}}}{spaceafter}
	%\tableentry{Phone}{+1 (617) 253 1000 x5322 (Work)}{}
	%\tableentry{Mobile}{+1 (232) 842-3583}{}
	
	%\\ % Additional vertical whitespace between the references
	
	%\tableentry{}{\textbf{Dr. Eli Vance}}{spaceafter}
	%\tableentry{Position}{Scientist (HL1)}{}
	%\tableentry{Employer}{\href{http://www.bmrf.us}{Black Mesa Research Facility}}{spaceafter}
	%\tableentry{Email}{\href{mailto:e.vance@bmrf.us}{e.vance@bmrf.us}}{}
	%\tableentry{Phone}{+1 (800) 786-1410 x6235 (Work)}{}
	%\tableentry{Mobile}{+1 (201) 632-3901}{}
	
	%------------------------------------------------
	
%\end{supertabular}

\medskip % Extra vertical whitespace before the next section

%----------------------------------------------------------------------------------------

\switchcolumn % Switch to the second (right) column

%----------------------------------------------------------------------------------------
%	COLORED CONTACT DETAILS BOX
%----------------------------------------------------------------------------------------

\parbox[top][0.11\textheight][c]{\linewidth}{ % Box to hold the colored box; change the fixed height as needed to match the box to the left
	\colorbox{shade}{ % Create colored box and specify background color
		% Start a table with two columns, the table will ensure everything is
		% aligned
		\begin{supertabular}{@{\hspace{3pt}} p{0.05\linewidth} | p{0.775\linewidth}}
			\raisebox{-1pt}{\faHome} & Nordparkstraße 10b, 76829 Landau \\ % Address
			\raisebox{-1pt}{\faPhone} & +49 1525 3606062\\ % Phone number
			\raisebox{-1pt}{\small\faEnvelope} & \href{mailto:bissantz@uni-landau.de}{bissantz@uni-landau.de} \\ % Email address
			\raisebox{-1pt}{\small\faGithub} & \href{https://github.com/sbissantz}{https://github.com/sbissantz} \\ % Website
			%\raisebox{-1pt}{\faGithub} & \href{https://github.com/username}{https://github.com/username} \\ % GitHub profile
			%\raisebox{-1pt}{\faLinkedinSquare} & \href{https://www.linkedin.com/in/username}{https://www.linkedin.com/in/username} \\ % LinkedIn profile
			% See fontawesome.pdf in the Fonts folder for all icons you can use
		\end{supertabular}
	}
	\vfill % Push content to the top of the box
}

%----------------------------------------------------------------------------------------
%	EDUCATION
%----------------------------------------------------------------------------------------

\section{Education} 

%\qualificationentry
%	{} % Duration
%	{} % Degree
%	{} % Honors, achievements or distinctions (e.g. first class honors)
%	{} % Department
%	{} % Institution

% All 5 parameters must be supplied but any can be empty if you don't need them

%------------------------------------------------

% Start a aligned two-column table
\begin{supertabular}{r l} 

%------------------------------------------------

	\qualificationentry
		{2019-2021}
		{M.\,A. Social Science \textsc{(1,0)}}
		{} 
		{Department of Social Sciences}
		{University of Koblenz-Landau}

	\qualificationentry
		{2018-2019}
		{Survey Statistics}
		{} 
		{Department of Economics} 
		{University of Bamberg}

	\qualificationentry
		{2014-2018}
		{B.\,A. Sociology \textsc{(1,6)}}
		{} 
		{Department of Sociology}
		{University of Mannheim}

%------------------------------------------------

\end{supertabular}

%----------------------------------------------------------------------------------------
%	AWARDS
%----------------------------------------------------------------------------------------

%\section{Awards}

%\tableentry{Heading}{Content}{spaceafter}
% All 3 parameters must be supplied but any can be empty if you don't need them
% A "spaceafter" value in the third parameter will add some vertical space --
% this is to be used between headings, leave it empty for no extra space

% Start a aligned two-column table
%\begin{supertabular}{r l}
	
	%\tableentry{1985}{\textbf{Faculty of Science Masters Scholarship}}{}
	%\tableentry{}{\textit{Massachusetts Institute of Technology}}{spaceafter}
	
	%\tableentry{1983}{\textbf{Top Achiever Award -- Physics}}{}
	%\tableentry{}{\textit{The University of Washington}}{spaceafter}
	
%\end{supertabular}

%----------------------------------------------------------------------------------------
%	PUBLICATIONS
%----------------------------------------------------------------------------------------

\section{Publications}

%------------------------------------------------

\textbf{Bißantz, S.} (2021). \textit{elisr}: Exploratory Likert Scaling in R. R
package version 0.2.0. \href{https://CRAN.R-project.org/package=elisr}{https://CRAN.R-project.org/pa\-ckage\-=\-elisr}

\medskip % Vertical whitespace

Wagner, I., Loesche, P. \& \textbf{Bißantz, S.} (2021). Low-stakes performance
testing in Germany by the \textsc{VERA} assessment: analysis of the mode
effects between computer-based test\-ing and paper-pencil testing. European
Journal of Psychology of Education. 10.1007/s10212021-00532-6

%------------------------------------------------

% As an alternative to a long-form publication list, you can create a shorter
% summary using only DOI values and years.

% Example \doipublication{} command to add another publication:

%\doipublication{Year}{DOI}{firstauthor}{spaceafter}

% All four parameters are required (can be empty though) A value of
% "firstauthor" in the third parameter will output the DOI in bold A
% "spaceafter" value in the fourth parameter will add some vertical space --
% this is to be used between years

%------------------------------------------------

% \subsection{Publications by DOI}

%\begin{supertabular}{r l}
	
	%------------------------------------------------
	
	%\doipublication{1996}{10.1021/jp951483+}{firstauthor}{spaceafter}
	
	%------------------------------------------------
	
	%\doipublication{1990}{10.1139/p90-097}{firstauthor}{spaceafter}
	
	%------------------------------------------------
	
	%\doipublication{1986}{10.1139/v86-297}{}{}
	%\doipublication{}{10.1103/PhysRevA.34.2329}{}{spaceafter}
	
	%------------------------------------------------
	
	%& \textit{First author publications in} \textbf{bold}\\
	
	%------------------------------------------------
	
%\end{supertabular}

% \medskip % Extra whitespace before the next section

%----------------------------------------------------------------------------------------
% Teaching Experience	
%----------------------------------------------------------------------------------------

\section{Teaching Experience}

%\tableentry{Heading}{Content}{spaceafter} All 3 parameters must be supplied
%but any can be empty if you don't need them A "spaceafter" value in the third
%parameter will add some vertical space -- this is to be used between headings,
%leave it empty for no extra space

%------------------------------------------------

\begin{supertabular}{r l}
	
	%------------------------------------------------
	
	\tableentry{2021 (WS)}{Test theory \& Diagnostics \hspace{1cm}\hfill \textsc{\footnotesize{(exercise)}}}{spaceafter}
	
	\tableentry{2022 (SS)}{Data analysis in R \hfill \textsc{\footnotesize{(exercise)}}}{spaceafter}

	\tableentry{2022 (SS)}{Test theory \& Diagnostics \hfill \textsc{\footnotesize{(seminar)}}}{spaceafter}
	
	%------------------------------------------------

\end{supertabular}

%------------------------------------------------------------------------------
%	COMPUTER SKILLS
%------------------------------------------------------------------------------

\section{Computer Skills} 

% This section is laid out using a table. A \tableentry command adds lines with the following parameters:

%\tableentry{Heading}{Content}{spaceafter}
% All 3 parameters must be supplied but any can be empty if you don't need them
% A "spaceafter" value in the third parameter will add some vertical space -- this is to be used between headings, leave it empty for no extra space

%------------------------------------------------

\begin{supertabular}{r l}
	
	%------------------------------------------------

	\tableentry{Operating Systems}{Linux, macOS \& Windows}{spaceafter}
	
	\tableentry{Linux Distributions}{Fedora, openSUSE \& Arch}{spaceafter}

	\tableentry{Statistical software}{R, Stan, Stata \& SPSS}{spaceafter}

	\tableentry{Miscellaneous}{Git, Vim \& Emacs}{spaceafter}

	%------------------------------------------------
	
\end{supertabular}

\textit{Proficiency: Server administrator in our working group
\enquote{Personality, Psychological Assessment, and Psychological Methods} at
the University of Koblenz-Landau}

%------------------------------------------------------------------------------
% Research Interests	
%------------------------------------------------------------------------------

\section{Research Interests}

\paragraph{Meta Science}
What is the effect of scale modifications on inferences, replicability and
effect-size heterogeneity? 

\vspace{-6pt}

\paragraph{Bayesian Data Analysis}
How to maximize information entropy using Bayesian updating?

\vspace{-6pt}

\paragraph{Maschine Learning}
How to minimize cross-entropy in prediction tasks?

% \subsection{Bayesian Data Analysis \& Maschine Learning}
% How to maximize information entropy using Bayesian updating
% or minimizing cross entropy in prediction tasks?

%----------------------------------------------------------------------------------------
%	SKILLS DESCRIPTION
%----------------------------------------------------------------------------------------

%\section{Skills}

%\subsection{Goal Oriented}

%I believe in action over long-winded discussions. I listen to everyone's
%viewpoints and use my judgement to immediately act based on consensus to
%achieve goals quickly and efficiently.

%\subsection{Physical Dexterity}

%Manual manipulation of experimental equipment and training within Black Mesa
%(e.g. the Hazard Course) have contributed to an enjoyment of working with my
%hands.

%\subsection{Passionate}

%I have been interested in theoretical physics such as quantum mechanics and
%relativity from an early age. My education and research have cemented this
%interest into a passion. I greatly enjoy carrying out fundamental physics
%research with potential practical applications.

%----------------------------------------------------------------------------------------

\end{paracol} % End two-column mode

%----------------------------------------------------------------------------------------

\end{document}
