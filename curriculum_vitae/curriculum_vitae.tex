%------------------------------------------------------------------------------
% A customized version of Vel's interpretation of the...
% Freeman Curriculum Vitae
% XeLaTeX Template
% Version 3.0 (September 3, 2021)
% see: https://www.LaTeXTemplates.com
% License:
% CC BY-NC-SA 4.0 (https://creativecommons.org/licenses/by-nc-sa/4.0/)
%------------------------------------------------------------------------------

%------------------------------------------------------------------------------
% Preamble
%------------------------------------------------------------------------------
\documentclass[10pt]{FreemanCV}

\usepackage{csquotes}

% Widths of the two columns, specified here as a ratio summing to 1 to
% correspond to percentages; adjust as needed for your content 
\columnratio{0.55, 0.45} 

% Headers and footers can be added with the following commands: \lhead{},
% \rhead{}, \lfoot{} and \rfoot{}
\lfoot{{\sffamily For the \LaTeX code and the pdf-Version, please see: \href{https://github.com/sbissantz/smip_application}{https://github.com/sbissantz}}}

%------------------------------------------------------------------------------

\begin{document}

\begin{paracol}{2} % Begin two-column mode

%------------------------------------------------------------------------------
%	YOUR NAME AND CURRICULUM VITAE TITLE
%----------------------------------------------------------------------------------------

\parbox[][0.11\textheight][c]{\linewidth}{ 
	% Box to hold your name and CV title; change the fixed height as needed to
	% match the colored box to the right
	\centering % Horizontally center text
	
	{\Huge\textcolor{headings}{Curriculum Vitae}} % Your name
	\par\noindent\rule{6cm}{0.4pt}
	\medskip % Vertical whitespace
	
	{\sffamily\Large{Steven Marcel Bißantz}}
	%{\cursivefont\Huge\textcolor{headings}{Curriculum Vitae}}
	
	\vfill % Push content to the top of the box
}

%------------------------------------------------------------------------------
%	MAJOR RESEARCH PROJECT
%------------------------------------------------------------------------------

\section{Doctoral Research}

{\raggedright\textbf{Measurement \& Replicability in the Sciences --
Meta-Scientific Investigation in The Role of Measurement in The Replicability
of Empirical Findings}\par}

\medskip % Vertical whitespace

My research examined the use of ELW pulses from a mode-locked source array
inducted through transuranic crystals to observe entanglement on supraquantum
structures. Theoretical advancements included prediction of quantum resonance
phenomena including the possibility of resonance cascades. I was motivated to
conduct this doctoral research due to my passion for teleportation of matter
and I believe I have laid the foundation for further experimental validation
and development of practical outcomes.

\medskip % Extra vertical whitespace before the next section

%------------------------------------------------------------------------------
%	WORK EXPERIENCE
%------------------------------------------------------------------------------

\section{Work Experience}

% Each job is added with a \jobentry command. Below is an empty one to use as a template:

%\jobentry
%	{} % Duration
%	{} % FT/PT (full time or part time)
%	{} % Employer
%	{} % Job title
%	{} % Description

% All 5 parameters must be supplied but any can be empty if you don't need them

%------------------------------------------------

\jobentry
	{Current, from Oct 2021} 
	{FT} 
	{University of Koblenz-Landau}
	{PhD Candidate, Psychology}  
	{As part of this description promotion, I began conducting nuclear and
		spacetime. The focus is on practical outcomes and applications in
		teleportation and communication with distal locations.} % Description

%------------------------------------------------

\jobentry
	{May 2019 -- Jul 2021}
	{PT} 
	{Centre of Educational Research (\textsc{ZEPF})}
	{Research Assistent, Psychology}  
	{Im Team forschen wir aktuell zu Moduseffekten in Vergleichsarbeiten mit
	Strukturgleichungs- und Mehrebenenmodellen.}

%------------------------------------------------

\jobentry
	{May 2019 -- Jul 2021}
	{PT} 
	{University of Mannheim}
	{Research Assistent, Sociology}  
	{Analyse- und Recherchearbeiten sowie die Verwaltung des Internetauftritts
	in englischer und deutscher Sprache zählten zu meinen Aufgaben.}

%------------------------------------------------

\jobentry
	{May 2019 -- Jul 2021}
	{PT} 
	{University of Mannheim}
	{Research Assistent, Sociology}  
	{Ich bereitete den Lehrstoff nach, klärte offene Fragen und vertiefte durch ergän-
	zende Lehreinheiten das Verständnis meiner Kommilitonen.}

%------------------------------------------------

\jobentry
	{May 2019 -- Jul 2021}
	{PT} 
	{Mannheim Centre for European Social Research (\textsc{MZES})}
	{Research Assistent, Sociology}  
	{Nach theoretischen und praktischen Schulungen führte ich com-
	putergestützte Telefoninterviews mit Jugendlichen durch.}

%----------------------------------------------------------------------------------------
% Memberships	
%----------------------------------------------------------------------------------------

\section{Memberships}

\jobentry
	{Current, from Feb 2021} 
	{} 
	{University of Koblenz-Landau}
	{Member of the Open Science Comission}  
	{As part of this description promotion, I began conducting nuclear and
		spacetime. The focus is on practical outcomes and applications in
		teleportation and communication with distal locations.} % Description

%----------------------------------------------------------------------------------------
%	REFERENCES
%----------------------------------------------------------------------------------------

\section{References}

%\textit{References available on request} % Uncomment if you'd rather not include references and remove the section below

%------------------------------------------------

% This section is laid out using a table. A \tableentry command adds lines with the following parameters:

%\tableentry{Heading}{Content}{spaceafter}
% All 3 parameters must be supplied but any can be empty if you don't need them
% A "spaceafter" value in the third parameter will add some vertical space -- this is to be used between headings, leave it empty for no extra space

%------------------------------------------------

\begin{supertabular}{r l} % Start a table with two columns, the table will ensure everything is aligned
	
	%------------------------------------------------
	
	\tableentry{}{\textbf{Dr. Isaac Kleiner}}{spaceafter}
	\tableentry{Position}{Professor}{}
	\tableentry{Employer}{\href{https://web.mit.edu/physics/}{Department of Physics}}{}
	\tableentry{}{\href{https://web.mit.edu}{\textit{Massachusetts Institute of Technology}}}{spaceafter}
	\tableentry{Phone}{+1 (617) 253 1000 x5322 (Work)}{}
	\tableentry{Mobile}{+1 (232) 842-3583}{}
	
	%------------------------------------------------
	
	\\ % Additional vertical whitespace between the references
	
	%------------------------------------------------
	
	\tableentry{}{\textbf{Dr. Eli Vance}}{spaceafter}
	\tableentry{Position}{Scientist (HL1)}{}
	\tableentry{Employer}{\href{http://www.bmrf.us}{Black Mesa Research Facility}}{spaceafter}
	\tableentry{Email}{\href{mailto:e.vance@bmrf.us}{e.vance@bmrf.us}}{}
	\tableentry{Phone}{+1 (800) 786-1410 x6235 (Work)}{}
	\tableentry{Mobile}{+1 (201) 632-3901}{}
	
	%------------------------------------------------
	
\end{supertabular}

\medskip % Extra vertical whitespace before the next section

%----------------------------------------------------------------------------------------

\switchcolumn % Switch to the second (right) column

%----------------------------------------------------------------------------------------
%	COLORED CONTACT DETAILS BOX
%----------------------------------------------------------------------------------------

\parbox[top][0.11\textheight][c]{\linewidth}{ % Box to hold the colored box; change the fixed height as needed to match the box to the left
	\colorbox{shade}{ % Create colored box and specify background color
		% Start a table with two columns, the table will ensure everything is
		% aligned
		\begin{supertabular}{@{\hspace{3pt}} p{0.05\linewidth} | p{0.775\linewidth}}
			\raisebox{-1pt}{\faHome} & Nordparkstraße 10b, 76829 Landau \\ % Address
			\raisebox{-1pt}{\faPhone} & +49 1525 3606062\\ % Phone number
			\raisebox{-1pt}{\small\faEnvelope} & \href{mailto:bissantz@uni-landau.de}{bissantz@uni-landau.de} \\ % Email address
			\raisebox{-1pt}{\small\faGithub} & \href{https://github.com/sbissantz}{https://github.com/sbissantz} \\ % Website
			%\raisebox{-1pt}{\faGithub} & \href{https://github.com/username}{https://github.com/username} \\ % GitHub profile
			%\raisebox{-1pt}{\faLinkedinSquare} & \href{https://www.linkedin.com/in/username}{https://www.linkedin.com/in/username} \\ % LinkedIn profile
			% See fontawesome.pdf in the Fonts folder for all icons you can use
		\end{supertabular}
	}
	\vfill % Push content to the top of the box
}

%----------------------------------------------------------------------------------------
%	EDUCATION
%----------------------------------------------------------------------------------------

\section{Education} 

% Each qualification entry is added with a \qualificationentry command. Below is an empty one to use as a template:

%\qualificationentry
%	{} % Duration
%	{} % Degree
%	{} % Honors, achievements or distinctions (e.g. first class honors)
%	{} % Department
%	{} % Institution

% All 5 parameters must be supplied but any can be empty if you don't need them

%------------------------------------------------

\begin{supertabular}{r l} % Start a table with two columns, the table will ensure everything is aligned

	%------------------------------------------------
	
	\qualificationentry
		{2019-2021} % Duration
		{Master of Arts, Social Sciences} % Degree
		{Final score: 1.0} % Honors, achievements or distinctions (e.g. first class honors)
		{Department of Social Sciences} % Department
		{University of Koblenz-Landau} % Institution
	
	%------------------------------------------------
	
	\qualificationentry
		{2014-2019} % Duration
		{Bachelor of Arts, Sociology} % Degree
		{Final score: 1.6} % Honors, achievements or distinctions (e.g. first class honors)
		{Department of Sociology} % Department
		{University of Mannheim} % Institution

\end{supertabular}

%----------------------------------------------------------------------------------------
%	AWARDS
%----------------------------------------------------------------------------------------

%\section{Awards}

% This section is laid out using a table. A \tableentry command adds lines with the following parameters:

%\tableentry{Heading}{Content}{spaceafter}
% All 3 parameters must be supplied but any can be empty if you don't need them
% A "spaceafter" value in the third parameter will add some vertical space -- this is to be used between headings, leave it empty for no extra space

%------------------------------------------------

%\begin{supertabular}{r l} % Start a table with two columns, the table will ensure everything is aligned
	
	%------------------------------------------------
	
	%\tableentry{1985}{\textbf{Faculty of Science Masters Scholarship}}{}
	%\tableentry{}{\textit{Massachusetts Institute of Technology}}{spaceafter}
	
	%------------------------------------------------
	
	%\tableentry{1983}{\textbf{Top Achiever Award -- Physics}}{}
	%\tableentry{}{\textit{The University of Washington}}{spaceafter}
	
	%------------------------------------------------
	
%\end{supertabular}

%----------------------------------------------------------------------------------------
%	PUBLICATIONS
%----------------------------------------------------------------------------------------

\section{Publications}

%------------------------------------------------

\textbf{Bißantz, S.} (2021). \textit{elisr}: Exploratory Likert Scaling in R.
R package version 0.2.0. \href{https://CRAN.R-project.org/package=elisr}{https://CRAN.R-project.org/pa\-ckage=elisr}

\medskip % Vertical whitespace

Wagner, I., Loesche, P. \& \textbf{Bißantz, S.} (2021). Low-stakes
performance testing in Germany by the VERA assessment:
analysis of the mode effects between computer-based test-
ing and paper-pencil testing. European Journal of Psycholo-
gy of Education. 10.1007/s10212021-00532-6

%------------------------------------------------

% As an alternative to a long-form publication list, you can create a shorter summary using only DOI values and years.

% Example \doipublication{} command to add another publication:

%\doipublication{Year}{DOI}{firstauthor}{spaceafter}

% All four parameters are required (can be empty though)
% A value of "firstauthor" in the third parameter will output the DOI in bold
% A "spaceafter" value in the fourth parameter will add some vertical space -- this is to be used between years

%------------------------------------------------

% \subsection{Publications by DOI}

%\begin{supertabular}{r l} % Start a table with two columns, the table will ensure everything is aligned
	
	%------------------------------------------------
	
	%\doipublication{1996}{10.1021/jp951483+}{firstauthor}{spaceafter}
	
	%------------------------------------------------
	
	%\doipublication{1990}{10.1139/p90-097}{firstauthor}{spaceafter}
	
	%------------------------------------------------
	
	%\doipublication{1986}{10.1139/v86-297}{}{}
	%\doipublication{}{10.1103/PhysRevA.34.2329}{}{spaceafter}
	
	%------------------------------------------------
	
	%& \textit{First author publications in} \textbf{bold}\\
	
	%------------------------------------------------
	
%\end{supertabular}

% \medskip % Extra whitespace before the next section

%----------------------------------------------------------------------------------------
% Teaching Experience	
%----------------------------------------------------------------------------------------

\section{Teaching Experience}

%\tableentry{Heading}{Content}{spaceafter} All 3 parameters must be supplied
%but any can be empty if you don't need them A "spaceafter" value in the third
%parameter will add some vertical space -- this is to be used between headings,
%leave it empty for no extra space

%------------------------------------------------

\begin{supertabular}{r l} % Start a table with two columns, the table will ensure everything is aligned
	
	%------------------------------------------------
	
	\tableentry{2021 (WS)}{Test theory \& Diagnostics (exercise)}{spaceafter}
	
	%------------------------------------------------
	
	\tableentry{2022 (SS)}{Data analysis with R (exercise)}{spaceafter}
	\tableentry{2022 (SS)}{Test theory \& Diagnostics (seminar) }{spaceafter}
	
	%------------------------------------------------

\end{supertabular}

%------------------------------------------------------------------------------
%	COMPUTER SKILLS
%------------------------------------------------------------------------------

\section{Computer Skills} 

% This section is laid out using a table. A \tableentry command adds lines with the following parameters:

%\tableentry{Heading}{Content}{spaceafter}
% All 3 parameters must be supplied but any can be empty if you don't need them
% A "spaceafter" value in the third parameter will add some vertical space -- this is to be used between headings, leave it empty for no extra space

%------------------------------------------------

\begin{supertabular}{r l} % Start a table with two columns, the table will ensure everything is aligned
	
	%------------------------------------------------

	\tableentry{Operating Systems}{Linux, macOS \& Windows}{spaceafter}
	
	\tableentry{Linux Distributions}{Fedora, Arch \& Debian}{spaceafter}

	\tableentry{Statistical software}{R, Stan, Stata \& SPSS}{spaceafter}

	\tableentry{Miscellaneous}{Git, Vim \& Emacs}{}

	%------------------------------------------------
	
\end{supertabular}

%------------------------------------------------------------------------------
% Research Interessts	
%------------------------------------------------------------------------------

\section{Research Interessts}

\paragraph{Meta Science}
What is the effect of scale modifications on inferences, replicability and
effect-size heterogeneity?

\subsection{Bayesian Data Analysis \& Maschine Learning}
How to maximize information entropy using Bayesian updating
or minimizing cross entropy in prediction tasks?

\subsection{Causal Inference}
Is there a way to combine coplex models, like a deep neural net, and causal
inference?

%----------------------------------------------------------------------------------------
%	SKILLS DESCRIPTION
%----------------------------------------------------------------------------------------

\section{Skills}

\subsection{Goal Oriented}

I believe in action over long-winded discussions. I listen to everyone's
viewpoints and use my judgement to immediately act based on consensus to
achieve goals quickly and efficiently.

\subsection{Physical Dexterity}

Manual manipulation of experimental equipment and training within Black Mesa
(e.g. the Hazard Course) have contributed to an enjoyment of working with my
hands.

\subsection{Passionate}

I have been interested in theoretical physics such as quantum mechanics and
relativity from an early age. My education and research have cemented this
interest into a passion. I greatly enjoy carrying out fundamental physics
research with potential practical applications.

%----------------------------------------------------------------------------------------

\end{paracol} % End two-column mode

%----------------------------------------------------------------------------------------

\end{document}
